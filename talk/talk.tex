\documentclass{beamer}
\usetheme[pageofpages=of,% String used between the current page and the
                         % total page count.
          bullet=circle,% Use circles instead of squares for bullets.
          titleline=true,% Show a line below the frame title.
          alternativetitlepage=true,% Use the fancy title page.
       %   titlepagelogo=logo-polito,% Logo for the first page.
       %   watermark=watermark-polito,% Watermark used in every page.
       %   watermarkheight=100px,% Height of the watermark.
       %   watermarkheightmult=4,% The watermark image is 4 times bigger
                                % than watermarkheight.
          ]{Torino}

\usecolortheme{dolphin} 
\usefonttheme{professionalfonts}

\setbeamertemplate{footline}{
  \begin{beamercolorbox}[wd=\paperwidth,ht=1ex,dp=1ex]{footline}
    \vspace{5pt} \hspace{1em} \insertframenumber/\inserttotalframenumber
  \end{beamercolorbox}
}

\author{Brendon J. Brewer}
\title{Diffusive Nested Sampling: Not Dead Yet}
\institute{Department of Statistics, The University of Auckland}
\date{}


\linespread{1.3}
\usepackage{minted}
\usepackage[utf8]{inputenc}
\usepackage{dsfont}

\newcommand{\given}{\,|\,}


\begin{document}

\frame{\titlepage}

\begin{frame}
\frametitle{Bayesian Inference}
Bayesian inference uses probability theory (usually probability distributions)
to describe uncertainty, not variability. Bayes' theorem is used frequently:
\pause
\begin{align}
p(\theta \given D, M) &= \frac{p(\theta \given M)p(D \given \theta, M)}{P(D \given M)}
\end{align}
\pause
where
\begin{align}
p(D \given M) &= \int p(\theta \given M)p(D \given \theta, M) \, d\theta
\end{align}


\end{frame}



\begin{frame}
\frametitle{Bayesian Inference}
The equations in words:

\begin{align}
\textnormal{posterior distribution} &= 
    \frac{\textnormal{prior distribution} \times \textnormal{likelihood function}}{\textnormal{marginal likelihood value}}
\end{align}
\pause
where
\begin{align}
\textnormal{marginal likelihood value} &= \textnormal{integral of prior times likelihood}.
\end{align}


\end{frame}


\begin{frame}
\frametitle{Bayesian Inference}
The choice of the {\bf prior distribution} and the {\bf likelihood} form the
inputs to the calculation. The former involves judgement calls, and the latter
involves judgement calls and the data.\\[0.5em]\pause

The {\bf posterior distribution} and
{\bf marginal likelihood value} are the outputs. But how can we calculate them
in general?

\end{frame}


\begin{frame}
\frametitle{Posterior Only}
Many Bayesians use Markov Chain Monte Carlo (MCMC) to generate {\bf samples}
from the posterior $p(\theta \given D, M)$, describing uncertainty about the
parameters that remains after incorporating the data.\\[0.5em]\pause

It is less common to compute the marginal likelihood, which is used for
model comparison purposes. E.g. for two competing models $M_1$ and $M_2$
the marginal likelihood acts as a likelihood:

\begin{align}
P(M_1 \given D) &= \frac{P(M_1)P(D \given M_1)}
                        {P(M_1)P(D \given M_1) + P(M_2)P(D \given M_2)}.
\end{align}

\end{frame}


\begin{frame}
\frametitle{Notation Switch}
For computational discussions:

\begin{align}
\pi(\theta) &= \textnormal{prior} \\
L(\theta) &= \textnormal{likelihood} \\
Z &= \textnormal{marginal likelihood}.
\end{align}

\end{frame}


\begin{frame}
\frametitle{Thermodynamic Integration}
An older technique to calculate $Z$ is based on the following
result:
\begin{align}
\log Z &= \int_0^1 \left<\log L(\theta)\right>_\beta \, d\theta
\end{align}
\pause
where the expectation is taken with respect to `power posteriors'
proportional to $\pi(\theta)L(\theta)^\beta$. \\[0.5em]\pause

An `annealing' process can be done
where $\beta$ is gradually raised from 0 to 1. As a side effect, difficult
posteriors such as those with multiple peaks become a bit easier.

\end{frame}


\begin{frame}
\frametitle{Thermodynamic Integration}
There are many methods based on the power posterior idea:

\begin{itemize}
\item Parallel tempering
\item Stepping-stone sampling
\item Sequential Monte Carlo
\item Simulated Tempering
\item More.
\end{itemize}

\end{frame}

\begin{frame}
\frametitle{Power Posteriors: Problems}
There is a class of problems that defeats methods based on power posteriors.
These are {\bf phase change problems}. In statistical physics they are common,
but they also occur in Bayesian inference --- they involve a particular
shape to the likelihood function.


\end{frame}


\begin{frame}
\frametitle{Phase Change Problems}

Let's explore how thermodynamic integration fails on phase change problems.


\end{frame}


\end{document}

